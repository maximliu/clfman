\label{problemstatement}
\begin{table*}[!t]
\centering
\caption{A Table on Specific FM Approaches to Different NFV Layers}
\label{tbl:layers}
	\begin{tabular}{c|p{3cm}|p{5cm}}
		\hline
		\textbf{Layer} & \textbf{Typical HA Approaches} & \textbf{Example}\\
		\hline
		\hline
		Physical	&	Heartbeat, Fencing, Hot/Cold Standby, Reliable Messaging Bus &
LinuxHA Project, Pacemaker, cman \\
		Hypervisor & Virtual Resource Monitoring, Live Migration & VMWare HA, HA-Lizard \\
		Virtual Network Functions & Application-specific FM methods & vendor
specific application \\
		Virtual Services & Traditional Network Service OSS/BSS & SNMP or other
vendor specific methods to manage faults of virtual devices \\
\hline
	\end{tabular}
\end{table*} 
 
The central problem in the cross-layer fault management lays in the coordination
of various FM appraoches from different layers involved in the NFV architecture
and, frequently, also within individual layers. We classify those problem in
\emph{vertical} and \emph{horizontal} dimensions. In this section, we analyze
research challenges in details accrodingly to the above mentioned dimensions,
then we provide a scenario that clarify an uncoordinated FM will aggreviate the
holistic system when failures impact. 

\subsection{Problems of Vertical Integration of FM Approaches} 

Typically each layer of NFV architecture possesses its own specific FM
approaches to ensure the services offered on this particular layer. There is
hardly any information exchanged between layers regarding FM and let alone the
coordination of separated layers. We argue that such a separate approach does
not meet requirements of high efficient overall FM system.  

\subsubsection{FM between Virtualization and Physical Layers}

\subsubsection{FM between NFVI and NFV Layers}

\subsubsection{FM by Orchestration of NFV Services}

\subsection{Problems of Horizontal Integration of FM Approaches}

\subsubsection{Coordinations between Distributed Clusters}

\subsubsection{Multi-Orchestrator Collaboration}
