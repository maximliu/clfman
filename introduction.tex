\label{intro}
Thank its flexibility which directly leads to significant savings on OPEX \&
CAPEX as well as its increased agility on service innovations, the concept of
NFV~\cite{nfv} has been prevailing recently among teleco service operators. The
fundamental idea behinds NFV is to build virtualized software network devices
based on the hypervisor and Commercial-Off-The-Shelf (COTS) hardware. The
infrastructure on which NFV is based is thus not built specifically for high
reliability and availability purposes; on the other hand, teleco services normally
requires five nines of availability which is about an annual outage time of 5
minutes. Providers are now facing with the dilemma: while enjoying high
flexibility and saving costs, they have to accept the risks of reliability. In
the context of NFV, \textit{faults have to be taken as facts rather than
exceptions}. Currently, this is also one of the major reasons which hold back
the teleco operators to adopt NFV system in their operational environment.

To tackle this dilemma, a highly efficient fault management system is needed to
deal with and compensate the un-reliable soft- and hardware systems.  As
prescribed by many literature on reliability and availability~\cite{depdef}, FM
including detection of system faults and toleration as well as isolation of
system anomalies. Finally impacted system should be recovered and services
restored in a timely manner. One may argue however that each layer involved in
the holistic NFV system already possesses its own mechanism to deal with fault,
however a fundamental question in this case one should rise: is it enough to
have a collection of individual FM systems to make a holistically efficient NFV
FM system? If not, what can be done about it? In this paper, we elaborate to
analyze and clarify the problematics.  The purpose of this paper is to identify
gaps and research challenges of building of highly efficient NFV FM system. This
paper is organized as follows: Section \ref{scenario} presents a typical NFV
scenario, on which our further discussions are based; in
Section~\ref{problemstatement} we make a in-depth analysis of the issues and
problems involved in FM of NVF and try to crystallize research challenges; in
Section~\ref{research} we discuss research issues and a roadmap to solve this
complex problem of cross-layer FM.  
